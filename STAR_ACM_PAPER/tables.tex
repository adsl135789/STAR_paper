% 主要實驗結果表格
\begin{table*}[t]
\centering
\small
\renewcommand{\arraystretch}{1.05}
\begin{tabular}{
    l
    @{\hskip 8pt}
    c@{\hskip 4pt}c@{\hskip 4pt}c
    @{\hskip 8pt}
    c@{\hskip 4pt}c@{\hskip 4pt}c
    @{\hskip 8pt}
    c@{\hskip 4pt}c@{\hskip 4pt}c
    @{\hskip 8pt}
    c@{\hskip 4pt}c@{\hskip 4pt}c
    @{\hskip 8pt}
    c@{\hskip 4pt}c@{\hskip 4pt}c
}
\toprule
\multirow{2}{*}[-1ex]{\textbf{Method}} &
\multicolumn{3}{c@{\hskip 8pt}}{\textbf{Mimo (ch)}} &
\multicolumn{3}{c@{\hskip 8pt}}{\textbf{Mimo (en)}} &
\multicolumn{3}{c@{\hskip 8pt}}{\textbf{OTTQA}} &
\multicolumn{3}{c@{\hskip 8pt}}{\textbf{FetaQA}} &
\multicolumn{3}{c}{\textbf{E2E-WTQ}} \\
\cmidrule(lr){2-4}
\cmidrule(lr){5-7}
\cmidrule(lr){8-10}
\cmidrule(lr){11-13}
\cmidrule(l){14-16}
& \textbf{R@1} & \textbf{R@5} & \textbf{R@10} & \textbf{R@1} & \textbf{R@5} & \textbf{R@10} & \textbf{R@1} & \textbf{R@5} & \textbf{R@10} & \textbf{R@1} & \textbf{R@5} & \textbf{R@10} & \textbf{R@1} & \textbf{R@5} & \textbf{R@10} \\
\midrule
QGpT & 49.81 & 71.06 & 77.23 & 50.66 & 72.35 & 80.80 & \uline{54.45} & 78.14 & 86.68 & 33.95 & 50.87 & 57.86 & 41.49 & 71.06 & 72.61 \\
STAR w/ FWF (0.7) & \uline{51.36} & \uline{72.16} & \uline{78.08} & \uline{58.34} & \uline{76.98} & \uline{82.50} & 53.84 & \uline{80.17} & \uline{88.17} & \uline{36.00} & \uline{54.92} & \uline{62.21} & \uline{58.51} & \uline{85.89} & \uline{90.04} \\
STAR w/ DWF & \textbf{51.58} & \textbf{72.15} & \textbf{77.99} & \textbf{58.89} & \textbf{77.72} & \textbf{82.89} & \textbf{54.07} & \textbf{79.99} & \textbf{88.08} & \textbf{36.25} & \textbf{54.77} & \textbf{62.21} & \textbf{58.51} & \textbf{85.06} & \textbf{90.06} \\
\bottomrule
\end{tabular}
\caption{主要實驗結果:STAR與QGpT在五個資料集上的比較。粗體表示最佳結果,底線表示次佳結果。}
\label{tab:main_results}
\end{table*}


% Ablation Study 表格 1
\begin{table}[t]
\centering
\small
\renewcommand{\arraystretch}{1.1}
\begin{tabular}{lccc}
\toprule
\textbf{Method} & \textbf{R@1} & \textbf{R@5} & \textbf{R@10} \\
\midrule
STAR (full) & \textbf{51.86} & \textbf{73.94} & \textbf{80.25} \\
w/o SCDG & 47.07 & 68.43 & 75.88 \\
w/o WF & 49.08 & 69.69 & 77.02 \\
QGpT & 45.47 & 67.68 & 75.04 \\
\bottomrule
\end{tabular}
\caption{消融實驗:STAR各個模組的有效性分析(五個資料集的平均Recall)。SCDG: Semantic Clustering and Query Generation;WF: Weighted Fusion。}
\label{tab:ablation}
\end{table}


% Ablation Study 表格 2
\begin{table}[t]
\centering
\small
\renewcommand{\arraystretch}{1.1}
\begin{tabular}{lccc}
\toprule
\textbf{Method} & \textbf{R@1} & \textbf{R@5} & \textbf{R@10} \\
\midrule
STAR w/ FWF (0.1) & 46.61 & 68.76 & 76.22 \\
STAR w/ FWF (0.3) & 49.25 & 71.66 & 78.65 \\
STAR w/ FWF (0.5) & 51.33 & 73.27 & 79.81 \\
STAR w/ FWF (0.7) & 51.62 & \uline{74.02} & \uline{80.20} \\
STAR w/ FWF (0.9) & 48.61 & 72.22 & 79.28 \\
STAR w/ DWF & \textbf{51.86} & \textbf{73.94} & \textbf{80.25} \\
\bottomrule
\end{tabular}
\caption{權重策略分析:不同Fixed Weight Fusion權重設置與Dynamic Weight Fusion的比較(五個資料集的平均Recall)。括號中的數字表示table的權重$\alpha$。}
\label{tab:weight_analysis}
\end{table}
