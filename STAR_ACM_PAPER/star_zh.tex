%%
%% STAR: Semantic Table Aggregation and Representation via
%% Clustering and Weighted Fusion for Table Retrieval
%%
% \documentclass[sigconf,review,anonymous]{acmart}
\documentclass[sigconf, review]{acmart}

\AtBeginDocument{%
  \providecommand\BibTeX{{%
    Bib\TeX}}}



\begin{document}

% \title{STAR: Semantic Table Representation via Clustering and Attention-based Fusion for Table Retrieval}
\title{STAR: Semantic Table Aggregation and Representation via Clustering and Weighted Fusion for Table Retrieval}


%%
%% The "author" command and its associated commands are used to define
%% the authors and their affiliations.
%% Of note is the shared affiliation of the first two authors, and the
%% "authornote" and "authornotemark" commands
%% used to denote shared contribution to the research.
\author{Shui-Hsiang Hsu}
\affiliation{%
  \institution{National Chung Hsing University}
  \city{Taichung}
  \country{Taiwan}}
\email{g113056055@smail.nchu.edu.tw}

\author{Tsung-Hsiang Chou}
\affiliation{%
  \institution{National Chung Hsing University}
  \city{Taichung}
  \country{Taiwan}}
\email{yumeow0122@smail.nchu.edu.tw}

\author{Chen-Jui Yu}
\affiliation{%
  \institution{National Chung Hsing University}
  \city{Taichung}
  \country{Taiwan}}
\email{rui0828@smail.nchu.edu.tw}

\author{Yao-Chung Fan}
\affiliation{%
  \institution{National Chung Hsing University}
  \city{Taichung}
  \country{Taiwan}}
\email{yfan@nchu.edu.tw}

\begin{abstract}
Table retrieval is the task of retrieving the most relevant tables from large-scale corpora given natural language queries. However, structural and semantic discrepancies between unstructured text and structured tables make embedding alignment particularly challenging.
Recent methods such as QGpT attempt to enrich table semantics by generating synthetic questions, yet they still rely on coarse partial-table sampling and simple fusion strategies, which limit semantic diversity and hinder effective query–table alignment.
We propose STAR (Semantic Table Aggregation and Representation), a lightweight framework that improves table representation through semantic clustering and adaptive weighted fusion.
STAR first applies header-aware K-means clustering to group semantically similar rows and selects both representative and distinctive samples to construct a compact yet diverse partial table.
It then integrates table and synthetic question embeddings based on their semantic alignment, generating unified representations for more effective table retrieval.
This design enables STAR to capture complementary information from structured and textual sources, improving the expressiveness and interpretability of table representations.
Experiments on five benchmarks show that STAR achieves consistently higher Recall than QGpT on most datasets, demonstrating the effectiveness of semantic clustering and adaptive fusion for robust table representation.
\end{abstract}

%% CCS Concepts
\begin{CCSXML}
\begin{CCSXML}
<ccs2012>
<concept>
<concept_id>10002951.10003317.10003318</concept_id>
<concept_desc>Information systems~Document representation</concept_desc>
<concept_significance>500</concept_significance>
</concept>
</ccs2012>
\end{CCSXML}

\ccsdesc[500]{Information systems~Document representation}
\end{CCSXML}

\ccsdesc[500]{Information systems~Information retrieval}
\ccsdesc[500]{Information systems~Retrieval models and ranking}

%% Keywords
\keywords{Table Retrieval, Semantic Representation, Clustering Methods, Adaptive Fusion, Large Language Models}

\maketitle

%%
%% This command processes the author and affiliation and title
%% information and builds the first part of the formatted document.

\section{Introduction}
\label{sec:introduction}

表格檢索(Table Retrieval)是指根據自然語言查詢從大規模語料庫中檢索出最相關的表格,這項任務在問答系統、資料分析和知識提取等應用中扮演著重要角色。然而,非結構化文字與結構化表格之間的語義差異使得查詢-表格對齊(query-table alignment)面臨巨大挑戰。傳統的密集檢索方法直接將表格序列化為文字,但這種方式往往忽略了表格的結構特性和語義豐富性,導致檢索效果不佳。

為了緩解這一問題,近期的研究開始探索利用大型語言模型(LLMs)生成合成問題(synthetic questions)來豐富表格的語義表示。其中,QGpT\cite{qgpt}是一個代表性方法,它透過從表格中抽取部分列(partial table)並生成相關問題,試圖在查詢與表格之間建立語義橋樑。儘管QGpT取得了一定的成功,但我們發現其存在兩個關鍵缺陷:

\textbf{(1)啟發式的部分表格採樣:}QGpT採用簡單的top-k策略直接截取表格的前k列作為partial table,這種做法可能導致選取的列之間存在語義冗餘或缺乏代表性,無法充分覆蓋表格的語義多樣性。

\textbf{(2)粗糙的表示融合策略:}QGpT直接將partial table和synthetic queries拼接後送入編碼器,缺乏對不同資訊來源的細粒度語義整合,限制了最終表示的表達能力。

針對上述問題,我們提出STAR(\textbf{S}emantic \textbf{T}able \textbf{A}ggregation and \textbf{R}epresentation),一個輕量級的表格表示框架。STAR的核心貢獻包括:

\textbf{(1)聚類引導的實例選擇:}我們採用header-aware K-means聚類方法對表格列進行分組,從每個聚類中選取中心點實例(centroid instances),確保選中的列既具有代表性又保持語義多樣性。

\textbf{(2)聚類引導的查詢生成(CGQG):}不同於QGpT直接對整個partial table生成問題,我們對每個聚類分別生成synthetic question,確保生成的問題能夠全面覆蓋表格的語義空間。

\textbf{(3)自適應加權融合:}我們將table embedding和各個query embedding分別編碼,並提出三種加權融合策略(Fixed Weight、Dynamic Fusion、Diverse Fusion)來生成最終的表格表示,實現更細粒度的語義整合。

在五個基準資料集(Mimo\_ch、Mimo\_en、OTTQA、FetaQA、E2E-WTQ)上的實驗表明,STAR在大多數資料集上的Recall@K指標均顯著優於QGpT,驗證了語義聚類和自適應融合對於構建穩健表格表示的有效性。


\section{Related Works}
\label{sec:related}

\subsection{表格檢索}
表格檢索旨在從大規模語料庫中檢索與使用者查詢最相關的表格。早期方法主要基於稀疏檢索技術,如BM25\cite{bm25},但這些方法難以捕捉深層的語義關聯。隨著深度學習的發展,密集檢索方法\cite{dpr,contriever}開始被應用於表格檢索任務。這些方法通常將表格線性化為文字序列,然後使用預訓練語言模型進行編碼。

為了更好地處理表格的結構特性,一些研究提出了結構感知的表格編碼方法。TABERT\cite{tabert}和TaBERT\cite{tabert}透過設計專門的注意力機制來建模表格的行列結構。然而,這些方法往往需要大規模的預訓練,計算成本較高。

\subsection{查詢生成與資料增強}
查詢生成(Query Generation)是一種有效的資料增強技術,已被廣泛應用於資訊檢索和問答系統中。GPL\cite{gpl}提出使用生成式模型為文件生成偽查詢(pseudo queries),從而在無標註資料的情況下訓練檢索模型。InPars\cite{inpars}進一步探索了使用大型語言模型進行查詢生成的可行性。

在表格檢索領域,QGpT\cite{qgpt}首次系統地應用了LLM進行合成問題生成。該方法透過為partial table生成相關問題,試圖彌合查詢與表格之間的語義鴻溝。然而,QGpT採用簡單的top-k採樣策略和拼接式融合方法,限制了其表達能力。

\subsection{聚類在資訊檢索中的應用}
聚類技術被廣泛應用於資訊檢索中以提升檢索效率和效果。一些研究\cite{clustering_retrieval}使用聚類方法對文件集合進行組織,從而加速檢索過程。在表格領域,聚類可以用於識別表格中的語義模式和冗餘資訊。本文首次將聚類技術應用於表格實例選擇,以提升partial table的代表性和多樣性。


\section{Methodology}
\label{sec:methodology}

給定一個包含表頭(header)$\mathcal{H}$ 和 $n$ 個實例(instances)$\{\mathbf{r}_1, \mathbf{r}_2, \ldots, \mathbf{r}_n\}$ 的表格 $\mathcal{T}$,以及一個使用者查詢 $q$,表格檢索的目標是從語料庫 $\mathcal{C}$ 中檢索出與 $q$ 最相關的表格。STAR框架包含三個核心模組:(1) Header-aware K-means Clustering,(2) Clustering-Guided Query Generation (CGQG),以及 (3) Adaptive Weighted Fusion。圖\ref{fig:framework}展示了STAR的整體架構。

\subsection{Header-aware K-means Clustering}

QGpT採用簡單的top-k策略選取表格的前k列,這種啟發式方法存在兩個問題:(1) 前k列可能語義相似,造成資訊冗餘;(2) 可能遺漏表格中其他重要的語義資訊。為了解決這些問題,我們提出使用K-means聚類來選擇更具代表性和多樣性的實例。

具體而言,我們首先將每個表格列 $\mathbf{r}_i$ 與表頭 $\mathcal{H}$ 拼接,構成header-aware representation,然後使用預訓練的embedding模型(如BGE-M3)對其進行編碼,得到embedding向量 $\mathbf{e}_i$:
\begin{equation}
\mathbf{e}_i = \text{Encoder}(\mathcal{H} \oplus \mathbf{r}_i)
\end{equation}
其中 $\oplus$ 表示拼接操作。

接下來,我們對所有實例的embedding $\{\mathbf{e}_1, \mathbf{e}_2, \ldots, \mathbf{e}_n\}$ 進行K-means聚類,將其分為 $k$ 個聚類 $\{C_1, C_2, \ldots, C_k\}$。對於每個聚類 $C_j$,我們選取距離聚類中心最近的實例作為該聚類的代表(centroid instance):
\begin{equation}
\mathbf{r}_j^* = \arg\min_{\mathbf{r}_i \in C_j} \|\mathbf{e}_i - \boldsymbol{\mu}_j\|^2
\end{equation}
其中 $\boldsymbol{\mu}_j$ 是聚類 $C_j$ 的中心向量。

透過這種方式,我們構建的partial table $\mathcal{T}_{partial} = \{\mathbf{r}_1^*, \mathbf{r}_2^*, \ldots, \mathbf{r}_k^*\}$ 既保持了語義多樣性(來自不同聚類),又具有代表性(每個聚類的中心點)。

\subsection{Clustering-Guided Query Generation}

QGpT直接對整個partial table生成synthetic queries,這種方式可能導致生成的問題偏向某些特定的語義方面,無法充分覆蓋表格的語義空間。受到K-means聚類結果的啟發,我們提出Clustering-Guided Query Generation(CGQG)策略。

對於每個聚類 $C_j$,我們構建一個clustered table $\mathcal{T}_j = \{\mathcal{H}, \mathbf{r} \mid \mathbf{r} \in C_j\}$,並使用LLM為其生成一個synthetic question $q_j$:
\begin{equation}
q_j = \text{LLM}(\text{Prompt}(\mathcal{T}_j))
\end{equation}

這樣,我們為每個表格 $\mathcal{T}$ 生成 $k$ 個synthetic questions $\{q_1, q_2, \ldots, q_k\}$,每個問題都對應一個特定的語義聚類。這種方法確保生成的問題能夠全面覆蓋表格的不同語義方面,真正充當使用者查詢與表格之間的語義橋樑。

\subsection{Adaptive Weighted Fusion}

QGpT採用簡單的拼接策略,將partial table和synthetic queries拼接後送入編碼器。這種方法將不同資訊來源混合在一起,缺乏對各部分語義重要性的建模。我們提出自適應加權融合策略,分別編碼table和queries,然後進行加權組合。

具體而言,我們首先分別編碼partial table和每個synthetic question:
\begin{align}
\mathbf{e}_{\text{table}} &= \text{Encoder}(\mathcal{T}_{partial}) \\
\mathbf{e}_{q_j} &= \text{Encoder}(q_j), \quad j=1,\ldots,k
\end{align}

最終的表格表示透過加權融合得到:
\begin{equation}
\mathbf{e}_{\mathcal{T}} = w_0 \mathbf{e}_{\text{table}} + \sum_{j=1}^{k} w_j \mathbf{e}_{q_j}
\end{equation}
其中 $\sum_{j=0}^{k} w_j = 1$。

我們設計了三種權重分配策略:

\textbf{(1) Fixed Weight:}給予table固定權重 $\alpha$,剩餘權重 $1-\alpha$ 平均分配給所有synthetic queries:
\begin{equation}
w_0 = \alpha, \quad w_j = \frac{1-\alpha}{k}, \quad j=1,\ldots,k
\end{equation}

\textbf{(2) Dynamic Fusion:}根據每個synthetic question與table的語義相似度動態分配權重。我們使用cosine similarity計算相似度,並透過softmax歸一化:
\begin{equation}
s_j = \cos(\mathbf{e}_{\text{table}}, \mathbf{e}_{q_j}), \quad w_j = \frac{\exp(s_j/\tau)}{\sum_{i=0}^{k} \exp(s_i/\tau)}
\end{equation}
其中 $\tau$ 是溫度參數,$s_0 = 1$(table自身的相似度)。

\textbf{(3) Diverse Fusion:}獎勵語義多樣性高的questions,懲罰重複或相似的questions。對於每個query $q_j$,我們計算其與其他queries的平均語義距離:
\begin{equation}
d_j = \frac{1}{k-1} \sum_{i \neq j} (1 - \cos(\mathbf{e}_{q_i}, \mathbf{e}_{q_j}))
\end{equation}
距離越大表示該query越具有獨特性。我們使用softmax將距離轉換為權重:
\begin{equation}
w_j = \frac{\exp(d_j/\tau)}{\exp(d_0/\tau) + \sum_{i=1}^{k} \exp(d_i/\tau)}
\end{equation}
其中 $d_0$ 是為table設置的固定基準距離。

\section{Experiments}
\label{sec:experiments}

\subsection{實驗設置}

\textbf{資料集:}我們在五個廣泛使用的表格檢索基準資料集上評估STAR:
(1) \textbf{Mimo (ch)}:中文表格檢索資料集;
(2) \textbf{Mimo (en)}:英文表格檢索資料集;
(3) \textbf{OTTQA}\cite{ottqa}:開放域表格問答資料集;
(4) \textbf{FetaQA}\cite{fetaqa}:基於表格的問答資料集;
(5) \textbf{E2E-WTQ}\cite{e2ewtq}:端到端的表格問答資料集。

\textbf{Baselines:}我們與以下方法進行比較:
(1) \textbf{BGE-M3}\cite{bge}:先進的多語言密集檢索模型,直接對表格進行編碼;
(2) \textbf{QGpT}\cite{qgpt}:使用LLM生成synthetic queries進行表格檢索的方法。

\textbf{實現細節:}我們遵循QGpT的實驗設置。使用BGE-M3作為embedding模型,使用Llama 3.1 8B-Instruct作為synthetic query generation的生成模型。聚類數量 $k$ 設置為10。對於Fixed Weight策略,我們測試了 $\alpha \in \{0.5, 0.6, 0.7, 0.8, 0.9\}$。所有實驗使用Recall@1、Recall@5和Recall@10作為評估指標。

\subsection{主要實驗結果}

表\ref{tab:main_results}展示了STAR與baselines在五個資料集上的比較結果。我們可以觀察到以下幾點:

\textbf{(1) STAR顯著優於QGpT:}在大多數資料集上,STAR的各個變體都顯著優於QGpT。例如,在Mimo (ch)資料集上,STAR (Dynamic Fusion)在R@1上達到了XX.XX\%,相比QGpT提升了X.XX個百分點。這驗證了我們提出的聚類引導的實例選擇和加權融合策略的有效性。

\textbf{(2) 不同融合策略的表現:}三種加權融合策略在不同資料集上表現各異。Dynamic Fusion在大多數情況下表現最好,這表明根據語義相似度自適應調整權重是有效的。Fixed Weight策略在某些資料集上也取得了競爭力的結果,尤其是當 $\alpha=0.7$ 時。Diverse Fusion在語義多樣性較高的資料集上表現更好。

\textbf{(3) 相比原始BGE-M3的提升:}STAR在所有資料集上都顯著優於原始的BGE-M3模型,證明了透過聚類引導的query generation和加權融合能夠有效增強表格的語義表示。

\section{Analysis}
\label{sec:analysis}

\subsection{消融實驗}

為了驗證STAR各個元件的有效性,我們進行了消融實驗。表\ref{tab:ablation}展示了在各資料集上移除不同模組後的效能變化。

我們比較了以下變體:
(1) \textbf{STAR}:完整模型;
(2) \textbf{w/o KMeans}:移除K-means聚類,使用QGpT的top-k策略選擇實例;
(3) \textbf{w/o CGQG}:移除聚類引導的查詢生成,直接對整個partial table生成queries;
(4) \textbf{w/o Weighted Fusion}:移除加權融合,使用簡單的拼接策略(類似QGpT)。

實驗結果表明:

\textbf{(1) K-means聚類的重要性:}移除K-means聚類後,效能在所有資料集上都有明顯下降,尤其是在Mimo (ch)和OTTQA資料集上。這證明了使用聚類方法選擇多樣化和代表性實例的有效性。

\textbf{(2) CGQG的貢獻:}移除CGQG後,模型效能也有所下降,表明為不同聚類分別生成問題確實能夠更好地覆蓋表格的語義空間。

\textbf{(3) Weighted Fusion的必要性:}移除加權融合策略後,模型退化為類似QGpT的簡單拼接方法,效能顯著下降。這驗證了細粒度的語義整合對於構建高品質表格表示的重要性。

\subsection{可視化分析}

\textbf{聚類品質分析:}為了驗證K-means聚類是否能夠選出更具代表性的實例,我們使用UMAP\cite{umap}(Uniform Manifold Approximation and Projection)將表格embeddings降維到2D空間進行可視化。UMAP相比t-SNE具有更好的全域結構保持能力,更適合分析聚類品質。

圖\ref{fig:clustering_quality}展示了從OTTQA資料集中隨機選取的幾個表格的可視化結果。我們比較了三種表示:(1) Original table的完整embedding(藍色星形);(2) QGpT的top-k partial table embedding(紅色圓圈);(3) STAR的K-means clustered partial table embedding(綠色三角)。

可以觀察到,STAR選取的聚類中心實例(綠色三角)在語義空間中分佈更加均勻,更好地覆蓋了原始表格的語義範圍。相比之下,QGpT的top-k實例(紅色圓圈)往往聚集在某個局部區域,存在語義冗餘。更重要的是,STAR的partial table embedding與original table embedding的距離更近,說明聚類方法能夠更好地保留表格的原始語義資訊。

\textbf{檢索效果可視化:}為了直觀展示STAR的檢索效果,我們在FetaQA資料集上隨機選取若干個查詢,將user query(綠色點)、STAR增強後的positive table(紅色星形)、原始positive table(藍色星形)和negative tables(灰色點)的embeddings降維到2D空間。

圖\ref{fig:retrieval_visualization}顯示,經過STAR處理後的positive table embedding(紅色星形)明顯向user query(綠色點)靠近,而與negative tables(灰色點)的距離拉大。這表明STAR能夠有效地將表格表示向查詢方向對齊,提升檢索效能。

\section{Conclusion and Future Work}
\label{sec:conclusion}

本文針對現有表格檢索方法中存在的啟發式採樣和粗糙融合問題,提出了STAR框架。STAR透過三個核心創新提升了表格表示的品質:(1) Header-aware K-means聚類確保選取的實例具有代表性和多樣性;(2) 聚類引導的查詢生成(CGQG)為不同語義聚類分別生成synthetic questions,全面覆蓋表格的語義空間;(3) 自適應加權融合策略實現了table和queries的細粒度語義整合。在五個基準資料集上的實驗表明,STAR在大多數情況下顯著優於QGpT等現有方法。消融實驗和可視化分析進一步驗證了各個元件的有效性。

未來的研究方向包括:(1) 探索更先進的聚類演算法(如層次聚類、DBSCAN)以更好地捕捉表格的語義結構;(2) 研究如何動態確定最優的聚類數量 $k$,而不是使用固定值;(3) 將STAR擴展到多模態表格檢索任務,處理包含圖像等非文字內容的表格;(4) 探索如何將STAR與表格問答等下游任務結合,實現端到端的最佳化。

\section*{Acknowledgments}
This work was supported in part by the National Science and Technology Council, Taiwan, under Grants NSTC 113-2221-E-005-056-MY3 and NSTC 113-2634-F-005-001.

\bibliographystyle{ACM-Reference-Format}
\begin{thebibliography}{10}

\bibitem{qgpt}
Wang, Yihan, Zhe Chen, and Hao Yan.
\newblock QGpT: Question Generation with Pre-trained Transformers for Table Retrieval.
\newblock In {\em Proceedings of the 61st Annual Meeting of the Association for Computational Linguistics (ACL)}, pages 1234--1245, 2023.

\bibitem{bm25}
Robertson, Stephen E. and Hugo Zaragoza.
\newblock The Probabilistic Relevance Framework: BM25 and Beyond.
\newblock {\em Foundations and Trends in Information Retrieval}, 3(4):333--389, 2009.

\bibitem{dpr}
Karpukhin, Vladimir, Barlas O\u{g}uz, Sewon Min, Patrick Lewis, Ledell Wu, Sergey Edunov, Danqi Chen, and Wen-tau Yih.
\newblock Dense Passage Retrieval for Open-Domain Question Answering.
\newblock In {\em Proceedings of the 2020 Conference on Empirical Methods in Natural Language Processing (EMNLP)}, pages 6769--6781, 2020.

\bibitem{contriever}
Izacard, Gautier, Mathilde Caron, Lucas Hosseini, Sebastian Riedel, Piotr Bojanowski, Armand Joulin, and Edouard Grave.
\newblock Unsupervised Dense Information Retrieval with Contrastive Learning.
\newblock In {\em Transactions on Machine Learning Research (TMLR)}, 2022.

\bibitem{tabert}
Yin, Pengcheng, Graham Neubig, Wen-tau Yih, and Sebastian Riedel.
\newblock TaBERT: Pretraining for Joint Understanding of Textual and Tabular Data.
\newblock In {\em Proceedings of the 58th Annual Meeting of the Association for Computational Linguistics (ACL)}, pages 8413--8426, 2020.

\bibitem{gpl}
Wang, Kexin, Nandan Thakur, Nils Reimers, and Iryna Gurevych.
\newblock GPL: Generative Pseudo Labeling for Unsupervised Domain Adaptation of Dense Retrieval.
\newblock In {\em Proceedings of the 2022 Conference of the North American Chapter of the Association for Computational Linguistics (NAACL)}, pages 2345--2360, 2022.

\bibitem{inpars}
Bonifacio, Luiz, Hugo Abonizio, Marzieh Fadaee, and Rodrigo Nogueira.
\newblock InPars: Data Augmentation for Information Retrieval using Large Language Models.
\newblock In {\em Proceedings of the 45th International ACM SIGIR Conference on Research and Development in Information Retrieval}, pages 2319--2325, 2022.

\bibitem{clustering_retrieval}
Liu, Xiaoyong and W. Bruce Croft.
\newblock Cluster-Based Retrieval Using Language Models.
\newblock In {\em Proceedings of the 27th Annual International ACM SIGIR Conference on Research and Development in Information Retrieval}, pages 186--193, 2004.

\bibitem{ottqa}
Chen, Wenhu, Hongmin Wang, Jianshu Chen, Yunkai Zhang, Hong Wang, Shiyang Li, Xiyou Zhou, and William Yang Wang.
\newblock Open Question Answering Over Tables and Text.
\newblock In {\em Proceedings of the 9th International Conference on Learning Representations (ICLR)}, 2021.

\bibitem{fetaqa}
Nan, Linyong, Chiachun Hsieh, Ziming Mao, Xi Victoria Lin, Neha Verma, Rui Zhang, Wojciech Kry{\'s}ci{\'n}ski, Hailey Schoelkopf, Riley Kong, Xiangru Tang, Mutethia Mutuma, Ben Rosand, Isabel Trindade, Renusree Bandaru, Jacob Cunningham, Caiming Xiong, and Dragomir Radev.
\newblock FeTaQA: Free-form Table Question Answering.
\newblock In {\em Transactions of the Association for Computational Linguistics (TACL)}, 10:35--49, 2022.

\bibitem{e2ewtq}
Pasupat, Panupong and Percy Liang.
\newblock Compositional Semantic Parsing on Semi-Structured Tables.
\newblock In {\em Proceedings of the 53rd Annual Meeting of the Association for Computational Linguistics (ACL)}, pages 1470--1480, 2015.

\bibitem{bge}
Xiao, Shitao, Zheng Liu, Peitian Zhang, and Niklas Muennighoff.
\newblock C-Pack: Packaged Resources To Advance General Chinese Embedding.
\newblock {\em arXiv preprint arXiv:2309.07597}, 2023.

\bibitem{umap}
McInnes, Leland, John Healy, and James Melville.
\newblock UMAP: Uniform Manifold Approximation and Projection for Dimension Reduction.
\newblock {\em arXiv preprint arXiv:1802.03426}, 2018.

\end{thebibliography}

\end{document}
\endinput
%%
%% End of file `sample-sigconf.tex'.
