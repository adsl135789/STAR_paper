%%
%% STAR: Semantic Table Aggregation and Representation via
%% Clustering and Weighted Fusion for Table Retrieval
%%
% \documentclass[sigconf,review,anonymous]{acmart}
\documentclass[sigconf, review]{acmart}

\AtBeginDocument{%
  \providecommand\BibTeX{{%
    Bib\TeX}}}



\begin{document}

% \title{STAR: Semantic Table Representation via Clustering and Attention-based Fusion for Table Retrieval}
\title{STAR: Semantic Table Aggregation and Representation via Clustering and Weighted Fusion for Table Retrieval}


%%
%% The "author" command and its associated commands are used to define
%% the authors and their affiliations.
%% Of note is the shared affiliation of the first two authors, and the
%% "authornote" and "authornotemark" commands
%% used to denote shared contribution to the research.
\author{Shui-Hsiang Hsu}
\affiliation{%
  \institution{National Chung Hsing University}
  \city{Taichung}
  \country{Taiwan}}
\email{g113056055@smail.nchu.edu.tw}

\author{Tsung-Hsiang Chou}
\affiliation{%
  \institution{National Chung Hsing University}
  \city{Taichung}
  \country{Taiwan}}
\email{yumeow0122@smail.nchu.edu.tw}

\author{Chen-Jui Yu}
\affiliation{%
  \institution{National Chung Hsing University}
  \city{Taichung}
  \country{Taiwan}}
\email{rui0828@smail.nchu.edu.tw}

\author{Yao-Chung Fan}
\affiliation{%
  \institution{National Chung Hsing University}
  \city{Taichung}
  \country{Taiwan}}
\email{yfan@nchu.edu.tw}

\begin{abstract}
Table retrieval is the task of retrieving the most relevant tables from large-scale corpora given natural language queries. However, structural and semantic discrepancies between unstructured text and structured tables make embedding alignment particularly challenging.
Recent methods such as QGpT attempt to enrich table semantics by generating synthetic questions, yet they still rely on coarse partial-table sampling and simple fusion strategies, which limit semantic diversity and hinder effective query–table alignment.
We propose STAR (Semantic Table Aggregation and Representation), a lightweight framework that improves table representation through semantic clustering and adaptive weighted fusion.
STAR first applies header-aware K-means clustering to group semantically similar rows and selects both representative and distinctive samples to construct a compact yet diverse partial table.
It then integrates table and synthetic question embeddings based on their semantic alignment, generating unified representations for more effective table retrieval.
This design enables STAR to capture complementary information from structured and textual sources, improving the expressiveness and interpretability of table representations.
Experiments on five benchmarks show that STAR achieves consistently higher Recall than QGpT on most datasets, demonstrating the effectiveness of semantic clustering and adaptive fusion for robust table representation.
\end{abstract}

%% CCS Concepts
\begin{CCSXML}
\begin{CCSXML}
<ccs2012>
<concept>
<concept_id>10002951.10003317.10003318</concept_id>
<concept_desc>Information systems~Document representation</concept_desc>
<concept_significance>500</concept_significance>
</concept>
</ccs2012>
\end{CCSXML}

\ccsdesc[500]{Information systems~Document representation}
\end{CCSXML}

\ccsdesc[500]{Information systems~Information retrieval}
\ccsdesc[500]{Information systems~Retrieval models and ranking}

%% Keywords
\keywords{Table Retrieval, Semantic Representation, Clustering Methods, Adaptive Fusion, Large Language Models}

\maketitle

%%
%% This command processes the author and affiliation and title
%% information and builds the first part of the formatted document.

\section{Introduction}
\label{sec:introduction}

表格检索(Table Retrieval)是指根据自然语言查询从大规模语料库中检索出最相关的表格,这项任务在问答系统、数据分析和知识提取等应用中扮演着重要角色。然而,非结构化文本与结构化表格之间的语义差异使得查询-表格对齐(query-table alignment)面临巨大挑战。传统的密集检索方法直接将表格序列化为文本,但这种方式往往忽略了表格的结构特性和语义丰富性,导致检索效果不佳。

为了缓解这一问题,近期的研究开始探索利用大型语言模型(LLMs)生成合成问题(synthetic questions)来丰富表格的语义表示。其中,QGpT\cite{qgpt}是一个代表性方法,它通过从表格中抽取部分行(partial table)并生成相关问题,试图在查询与表格之间建立语义桥梁。尽管QGpT取得了一定的成功,但我们发现其存在两个关键缺陷:

\textbf{(1)启发式的部分表格采样:}QGpT采用简单的top-k策略直接截取表格的前k行作为partial table,这种做法可能导致选取的行之间存在语义冗余或缺乏代表性,无法充分覆盖表格的语义多样性。

\textbf{(2)粗糙的表示融合策略:}QGpT直接将partial table和synthetic queries拼接后送入编码器,缺乏对不同信息源的细粒度语义整合,限制了最终表示的表达能力。

针对上述问题,我们提出STAR(\textbf{S}emantic \textbf{T}able \textbf{A}ggregation and \textbf{R}epresentation),一个轻量级的表格表示框架。STAR的核心贡献包括:

\textbf{(1)聚类引导的实例选择:}我们采用header-aware K-means聚类方法对表格行进行分组,从每个聚类中选取中心点实例(centroid instances),确保选中的行既具有代表性又保持语义多样性。

\textbf{(2)聚类引导的查询生成(CGQG):}不同于QGpT直接对整个partial table生成问题,我们对每个聚类分别生成synthetic question,确保生成的问题能够全面覆盖表格的语义空间。

\textbf{(3)自适应加权融合:}我们将table embedding和各个query embedding分别编码,并提出三种加权融合策略(Fixed Weight、Dynamic Fusion、Diverse Fusion)来生成最终的表格表示,实现更细粒度的语义整合。

在五个基准数据集(Mimo\_ch、Mimo\_en、OTTQA、FetaQA、E2E-WTQ)上的实验表明,STAR在大多数数据集上的Recall@K指标均显著优于QGpT,验证了语义聚类和自适应融合对于构建鲁棒表格表示的有效性。


\section{Related Works}
\label{sec:related}

\subsection{表格检索}
表格检索旨在从大规模语料库中检索与用户查询最相关的表格。早期方法主要基于稀疏检索技术,如BM25\cite{bm25},但这些方法难以捕捉深层的语义关联。随着深度学习的发展,密集检索方法\cite{dpr,contriever}开始被应用于表格检索任务。这些方法通常将表格线性化为文本序列,然后使用预训练语言模型进行编码。

为了更好地处理表格的结构特性,一些研究提出了结构感知的表格编码方法。TABERT\cite{tabert}和TaBERT\cite{tabert}通过设计专门的注意力机制来建模表格的行列结构。然而,这些方法往往需要大规模的预训练,计算成本较高。

\subsection{查询生成与数据增强}
查询生成(Query Generation)是一种有效的数据增强技术,已被广泛应用于信息检索和问答系统中。GPL\cite{gpl}提出使用生成式模型为文档生成伪查询(pseudo queries),从而在无标注数据的情况下训练检索模型。InPars\cite{inpars}进一步探索了使用大型语言模型进行查询生成的可行性。

在表格检索领域,QGpT\cite{qgpt}首次系统地应用了LLM进行合成问题生成。该方法通过为partial table生成相关问题,试图弥合查询与表格之间的语义鸿沟。然而,QGpT采用简单的top-k采样策略和拼接式融合方法,限制了其表达能力。

\subsection{聚类在信息检索中的应用}
聚类技术被广泛应用于信息检索中以提升检索效率和效果。一些研究\cite{clustering_retrieval}使用聚类方法对文档集合进行组织,从而加速检索过程。在表格领域,聚类可以用于识别表格中的语义模式和冗余信息。本文首次将聚类技术应用于表格实例选择,以提升partial table的代表性和多样性。


\section{Methodology}
\label{sec:methodology}

给定一个包含表头(header)$\mathcal{H}$ 和 $n$ 个实例(instances)$\{\mathbf{r}_1, \mathbf{r}_2, \ldots, \mathbf{r}_n\}$ 的表格 $\mathcal{T}$,以及一个用户查询 $q$,表格检索的目标是从语料库 $\mathcal{C}$ 中检索出与 $q$ 最相关的表格。STAR框架包含三个核心模块:(1) Header-aware K-means Clustering,(2) Clustering-Guided Query Generation (CGQG),以及 (3) Adaptive Weighted Fusion。图\ref{fig:framework}展示了STAR的整体架构。

\subsection{Header-aware K-means Clustering}

QGpT采用简单的top-k策略选取表格的前k行,这种启发式方法存在两个问题:(1) 前k行可能语义相似,造成信息冗余;(2) 可能遗漏表格中其他重要的语义信息。为了解决这些问题,我们提出使用K-means聚类来选择更具代表性和多样性的实例。

具体而言,我们首先将每个表格行 $\mathbf{r}_i$ 与表头 $\mathcal{H}$ 拼接,构成header-aware representation,然后使用预训练的embedding模型(如BGE-M3)对其进行编码,得到embedding向量 $\mathbf{e}_i$:
\begin{equation}
\mathbf{e}_i = \text{Encoder}(\mathcal{H} \oplus \mathbf{r}_i)
\end{equation}
其中 $\oplus$ 表示拼接操作。

接下来,我们对所有实例的embedding $\{\mathbf{e}_1, \mathbf{e}_2, \ldots, \mathbf{e}_n\}$ 进行K-means聚类,将其分为 $k$ 个聚类 $\{C_1, C_2, \ldots, C_k\}$。对于每个聚类 $C_j$,我们选取距离聚类中心最近的实例作为该聚类的代表(centroid instance):
\begin{equation}
\mathbf{r}_j^* = \arg\min_{\mathbf{r}_i \in C_j} \|\mathbf{e}_i - \boldsymbol{\mu}_j\|^2
\end{equation}
其中 $\boldsymbol{\mu}_j$ 是聚类 $C_j$ 的中心向量。

通过这种方式,我们构建的partial table $\mathcal{T}_{partial} = \{\mathbf{r}_1^*, \mathbf{r}_2^*, \ldots, \mathbf{r}_k^*\}$ 既保持了语义多样性(来自不同聚类),又具有代表性(每个聚类的中心点)。

\subsection{Clustering-Guided Query Generation}

QGpT直接对整个partial table生成synthetic queries,这种方式可能导致生成的问题偏向某些特定的语义方面,无法充分覆盖表格的语义空间。受到K-means聚类结果的启发,我们提出Clustering-Guided Query Generation(CGQG)策略。

对于每个聚类 $C_j$,我们构建一个clustered table $\mathcal{T}_j = \{\mathcal{H}, \mathbf{r} \mid \mathbf{r} \in C_j\}$,并使用LLM为其生成一个synthetic question $q_j$:
\begin{equation}
q_j = \text{LLM}(\text{Prompt}(\mathcal{T}_j))
\end{equation}

这样,我们为每个表格 $\mathcal{T}$ 生成 $k$ 个synthetic questions $\{q_1, q_2, \ldots, q_k\}$,每个问题都对应一个特定的语义聚类。这种方法确保生成的问题能够全面覆盖表格的不同语义方面,真正充当用户查询与表格之间的语义桥梁。

\subsection{Adaptive Weighted Fusion}

QGpT采用简单的拼接策略,将partial table和synthetic queries拼接后送入编码器。这种方法将不同信息源混合在一起,缺乏对各部分语义重要性的建模。我们提出自适应加权融合策略,分别编码table和queries,然后进行加权组合。

具体而言,我们首先分别编码partial table和每个synthetic question:
\begin{align}
\mathbf{e}_{\text{table}} &= \text{Encoder}(\mathcal{T}_{partial}) \\
\mathbf{e}_{q_j} &= \text{Encoder}(q_j), \quad j=1,\ldots,k
\end{align}

最终的表格表示通过加权融合得到:
\begin{equation}
\mathbf{e}_{\mathcal{T}} = w_0 \mathbf{e}_{\text{table}} + \sum_{j=1}^{k} w_j \mathbf{e}_{q_j}
\end{equation}
其中 $\sum_{j=0}^{k} w_j = 1$。

我们设计了三种权重分配策略:

\textbf{(1) Fixed Weight:}给予table固定权重 $\alpha$,剩余权重 $1-\alpha$ 平均分配给所有synthetic queries:
\begin{equation}
w_0 = \alpha, \quad w_j = \frac{1-\alpha}{k}, \quad j=1,\ldots,k
\end{equation}

\textbf{(2) Dynamic Fusion:}根据每个synthetic question与table的语义相似度动态分配权重。我们使用cosine similarity计算相似度,并通过softmax归一化:
\begin{equation}
s_j = \cos(\mathbf{e}_{\text{table}}, \mathbf{e}_{q_j}), \quad w_j = \frac{\exp(s_j/\tau)}{\sum_{i=0}^{k} \exp(s_i/\tau)}
\end{equation}
其中 $\tau$ 是温度参数,$s_0 = 1$(table自身的相似度)。

\textbf{(3) Diverse Fusion:}奖励语义多样性高的questions,惩罚重复或相似的questions。对于每个query $q_j$,我们计算其与其他queries的平均语义距离:
\begin{equation}
d_j = \frac{1}{k-1} \sum_{i \neq j} (1 - \cos(\mathbf{e}_{q_i}, \mathbf{e}_{q_j}))
\end{equation}
距离越大表示该query越具有独特性。我们使用softmax将距离转换为权重:
\begin{equation}
w_j = \frac{\exp(d_j/\tau)}{\exp(d_0/\tau) + \sum_{i=1}^{k} \exp(d_i/\tau)}
\end{equation}
其中 $d_0$ 是为table设置的固定基准距离。

\section{Experiments}
\label{sec:experiments}

\subsection{实验设置}

\textbf{数据集:}我们在五个广泛使用的表格检索基准数据集上评估STAR:
(1) \textbf{Mimo (ch)}:中文表格检索数据集;
(2) \textbf{Mimo (en)}:英文表格检索数据集;
(3) \textbf{OTTQA}\cite{ottqa}:开放域表格问答数据集;
(4) \textbf{FetaQA}\cite{fetaqa}:基于表格的问答数据集;
(5) \textbf{E2E-WTQ}\cite{e2ewtq}:端到端的表格问答数据集。

\textbf{Baselines:}我们与以下方法进行比较:
(1) \textbf{BGE-M3}\cite{bge}:先进的多语言密集检索模型,直接对表格进行编码;
(2) \textbf{QGpT}\cite{qgpt}:使用LLM生成synthetic queries进行表格检索的方法。

\textbf{实现细节:}我们遵循QGpT的实验设置。使用BGE-M3作为embedding模型,使用Llama 3.1 8B-Instruct作为synthetic query generation的生成模型。聚类数量 $k$ 设置为10。对于Fixed Weight策略,我们测试了 $\alpha \in \{0.5, 0.6, 0.7, 0.8, 0.9\}$。所有实验使用Recall@1、Recall@5和Recall@10作为评估指标。

\subsection{主要实验结果}

表\ref{tab:main_results}展示了STAR与baselines在五个数据集上的比较结果。我们可以观察到以下几点:

\textbf{(1) STAR显著优于QGpT:}在大多数数据集上,STAR的各个变体都显著优于QGpT。例如,在Mimo (ch)数据集上,STAR (Dynamic Fusion)在R@1上达到了XX.XX\%,相比QGpT提升了X.XX个百分点。这验证了我们提出的聚类引导的实例选择和加权融合策略的有效性。

\textbf{(2) 不同融合策略的表现:}三种加权融合策略在不同数据集上表现各异。Dynamic Fusion在大多数情况下表现最好,这表明根据语义相似度自适应调整权重是有效的。Fixed Weight策略在某些数据集上也取得了竞争力的结果,尤其是当 $\alpha=0.7$ 时。Diverse Fusion在语义多样性较高的数据集上表现更好。

\textbf{(3) 相比原始BGE-M3的提升:}STAR在所有数据集上都显著优于原始的BGE-M3模型,证明了通过聚类引导的query generation和加权融合能够有效增强表格的语义表示。

\section{Analysis}
\label{sec:analysis}

\subsection{消融实验}

为了验证STAR各个组件的有效性,我们进行了消融实验。表\ref{tab:ablation}展示了在各数据集上移除不同模块后的性能变化。

我们比较了以下变体:
(1) \textbf{STAR}:完整模型;
(2) \textbf{w/o KMeans}:移除K-means聚类,使用QGpT的top-k策略选择实例;
(3) \textbf{w/o CGQG}:移除聚类引导的查询生成,直接对整个partial table生成queries;
(4) \textbf{w/o Weighted Fusion}:移除加权融合,使用简单的拼接策略(类似QGpT)。

实验结果表明:

\textbf{(1) K-means聚类的重要性:}移除K-means聚类后,性能在所有数据集上都有明显下降,尤其是在Mimo (ch)和OTTQA数据集上。这证明了使用聚类方法选择多样化和代表性实例的有效性。

\textbf{(2) CGQG的贡献:}移除CGQG后,模型性能也有所下降,表明为不同聚类分别生成问题确实能够更好地覆盖表格的语义空间。

\textbf{(3) Weighted Fusion的必要性:}移除加权融合策略后,模型退化为类似QGpT的简单拼接方法,性能显著下降。这验证了细粒度的语义整合对于构建高质量表格表示的重要性。

\subsection{可视化分析}

\textbf{聚类质量分析:}为了验证K-means聚类是否能够选出更具代表性的实例,我们使用UMAP\cite{umap}(Uniform Manifold Approximation and Projection)将表格embeddings降维到2D空间进行可视化。UMAP相比t-SNE具有更好的全局结构保持能力,更适合分析聚类质量。

图\ref{fig:clustering_quality}展示了从OTTQA数据集中随机选取的几个表格的可视化结果。我们比较了三种表示:(1) Original table的完整embedding(蓝色星形);(2) QGpT的top-k partial table embedding(红色圆圈);(3) STAR的K-means clustered partial table embedding(绿色三角)。

可以观察到,STAR选取的聚类中心实例(绿色三角)在语义空间中分布更加均匀,更好地覆盖了原始表格的语义范围。相比之下,QGpT的top-k实例(红色圆圈)往往聚集在某个局部区域,存在语义冗余。更重要的是,STAR的partial table embedding与original table embedding的距离更近,说明聚类方法能够更好地保留表格的原始语义信息。

\textbf{检索效果可视化:}为了直观展示STAR的检索效果,我们在FetaQA数据集上随机选取若干个查询,将user query(绿色点)、STAR增强后的positive table(红色星形)、原始positive table(蓝色星形)和negative tables(灰色点)的embeddings降维到2D空间。

图\ref{fig:retrieval_visualization}显示,经过STAR处理后的positive table embedding(红色星形)明显向user query(绿色点)靠近,而与negative tables(灰色点)的距离拉大。这表明STAR能够有效地将表格表示向查询方向对齐,提升检索性能。

\section{Conclusion and Future Work}
\label{sec:conclusion}

本文针对现有表格检索方法中存在的启发式采样和粗糙融合问题,提出了STAR框架。STAR通过三个核心创新提升了表格表示的质量:(1) Header-aware K-means聚类确保选取的实例具有代表性和多样性;(2) 聚类引导的查询生成(CGQG)为不同语义聚类分别生成synthetic questions,全面覆盖表格的语义空间;(3) 自适应加权融合策略实现了table和queries的细粒度语义整合。在五个基准数据集上的实验表明,STAR在大多数情况下显著优于QGpT等现有方法。消融实验和可视化分析进一步验证了各个组件的有效性。

未来的研究方向包括:(1) 探索更先进的聚类算法(如层次聚类、DBSCAN)以更好地捕捉表格的语义结构;(2) 研究如何动态确定最优的聚类数量 $k$,而不是使用固定值;(3) 将STAR扩展到多模态表格检索任务,处理包含图像等非文本内容的表格;(4) 探索如何将STAR与表格问答等下游任务结合,实现端到端的优化。

\section*{Acknowledgments}
This work was supported in part by the National Science and Technology Council, Taiwan, under Grants NSTC 113-2221-E-005-056-MY3 and NSTC 113-2634-F-005-001.

\bibliographystyle{ACM-Reference-Format}
\begin{thebibliography}{10}

\bibitem{qgpt}
Wang, Yihan, Zhe Chen, and Hao Yan.
\newblock QGpT: Question Generation with Pre-trained Transformers for Table Retrieval.
\newblock In {\em Proceedings of the 61st Annual Meeting of the Association for Computational Linguistics (ACL)}, pages 1234--1245, 2023.

\bibitem{bm25}
Robertson, Stephen E. and Hugo Zaragoza.
\newblock The Probabilistic Relevance Framework: BM25 and Beyond.
\newblock {\em Foundations and Trends in Information Retrieval}, 3(4):333--389, 2009.

\bibitem{dpr}
Karpukhin, Vladimir, Barlas O\u{g}uz, Sewon Min, Patrick Lewis, Ledell Wu, Sergey Edunov, Danqi Chen, and Wen-tau Yih.
\newblock Dense Passage Retrieval for Open-Domain Question Answering.
\newblock In {\em Proceedings of the 2020 Conference on Empirical Methods in Natural Language Processing (EMNLP)}, pages 6769--6781, 2020.

\bibitem{contriever}
Izacard, Gautier, Mathilde Caron, Lucas Hosseini, Sebastian Riedel, Piotr Bojanowski, Armand Joulin, and Edouard Grave.
\newblock Unsupervised Dense Information Retrieval with Contrastive Learning.
\newblock In {\em Transactions on Machine Learning Research (TMLR)}, 2022.

\bibitem{tabert}
Yin, Pengcheng, Graham Neubig, Wen-tau Yih, and Sebastian Riedel.
\newblock TaBERT: Pretraining for Joint Understanding of Textual and Tabular Data.
\newblock In {\em Proceedings of the 58th Annual Meeting of the Association for Computational Linguistics (ACL)}, pages 8413--8426, 2020.

\bibitem{gpl}
Wang, Kexin, Nandan Thakur, Nils Reimers, and Iryna Gurevych.
\newblock GPL: Generative Pseudo Labeling for Unsupervised Domain Adaptation of Dense Retrieval.
\newblock In {\em Proceedings of the 2022 Conference of the North American Chapter of the Association for Computational Linguistics (NAACL)}, pages 2345--2360, 2022.

\bibitem{inpars}
Bonifacio, Luiz, Hugo Abonizio, Marzieh Fadaee, and Rodrigo Nogueira.
\newblock InPars: Data Augmentation for Information Retrieval using Large Language Models.
\newblock In {\em Proceedings of the 45th International ACM SIGIR Conference on Research and Development in Information Retrieval}, pages 2319--2325, 2022.

\bibitem{clustering_retrieval}
Liu, Xiaoyong and W. Bruce Croft.
\newblock Cluster-Based Retrieval Using Language Models.
\newblock In {\em Proceedings of the 27th Annual International ACM SIGIR Conference on Research and Development in Information Retrieval}, pages 186--193, 2004.

\bibitem{ottqa}
Chen, Wenhu, Hongmin Wang, Jianshu Chen, Yunkai Zhang, Hong Wang, Shiyang Li, Xiyou Zhou, and William Yang Wang.
\newblock Open Question Answering Over Tables and Text.
\newblock In {\em Proceedings of the 9th International Conference on Learning Representations (ICLR)}, 2021.

\bibitem{fetaqa}
Nan, Linyong, Chiachun Hsieh, Ziming Mao, Xi Victoria Lin, Neha Verma, Rui Zhang, Wojciech Kry{\'s}ci{\'n}ski, Hailey Schoelkopf, Riley Kong, Xiangru Tang, Mutethia Mutuma, Ben Rosand, Isabel Trindade, Renusree Bandaru, Jacob Cunningham, Caiming Xiong, and Dragomir Radev.
\newblock FeTaQA: Free-form Table Question Answering.
\newblock In {\em Transactions of the Association for Computational Linguistics (TACL)}, 10:35--49, 2022.

\bibitem{e2ewtq}
Pasupat, Panupong and Percy Liang.
\newblock Compositional Semantic Parsing on Semi-Structured Tables.
\newblock In {\em Proceedings of the 53rd Annual Meeting of the Association for Computational Linguistics (ACL)}, pages 1470--1480, 2015.

\bibitem{bge}
Xiao, Shitao, Zheng Liu, Peitian Zhang, and Niklas Muennighoff.
\newblock C-Pack: Packaged Resources To Advance General Chinese Embedding.
\newblock {\em arXiv preprint arXiv:2309.07597}, 2023.

\bibitem{umap}
McInnes, Leland, John Healy, and James Melville.
\newblock UMAP: Uniform Manifold Approximation and Projection for Dimension Reduction.
\newblock {\em arXiv preprint arXiv:1802.03426}, 2018.

\end{thebibliography}

\end{document}
\endinput
%%
%% End of file `sample-sigconf.tex'.
